\newacronym[description={Free and Open Source Software; software that grants the user the freedoms to use, study, share, and improve the program}]{foss}{FOSS}{Free and Open Source Software}

\newacronym{ip}{IP}{Intellectual Property}

\newacronym{fsf}{FSF}{Free Software Foundation}

\newacronym{fsfe}{FSFE}{Free Software Foundation Europe}

\newacronym[description={Software Package Data Exchange; an open standard for communicating software bill of material information}]{spdx}{SPDX}{Software Package Data Exchange}

\newacronym{ci}{CI}{Continuous Integration}

\newglossaryentry{spdx-license-identifier}{
    name={SPDX License Identifier},
    description={unique identifier for a \gls{foss} \gls{license}}
}

\newglossaryentry{copyright}{
    name={copyright},
    description={exclusive rights over original works of authorship}
}

\newglossaryentry{copyright-assignment}{
    name={copyright assignment},
    description={all third-party contributors to a project must sign an agreement wherein they transfer their \gls{copyright} to the first party}
}

\newglossaryentry{license}{
    name={license},
    description={the terms under which the \gls{copyright-holder} allows the recipient
    of the license to use the software}
}

\newglossaryentry{copyleft}{
    name={copyleft},
    description={a method for making a work free, and requiring
    all modified and extended versions of the work to be free as well}
}

\newglossaryentry{permissive}{
    name={permissive},
    description={when describing a \gls{license}: a non-\gls{copyleft} \gls{foss}
    \gls{license}---one that guarantees the freedoms to use, modify, and
    redistribute, but that permits \gls{proprietary} derivative works}
}

\newglossaryentry{proprietary}{
    name={proprietary},
    description={when describing a \gls{license}: a \gls{license} that does not grant the
    recipient the four essential freedoms as described by the Free Software
    Foundation}
}

\newglossaryentry{non-free}{
    name={non-free},
    description={see \gls{proprietary}}
}

\newglossaryentry{author}{
    name={author},
    description={creator of a work, often---but not always---the
    \gls{copyright-holder}}
}

\newglossaryentry{copyright-holder}{
    name={copyright holder},
    description={the rights holder to a work; sometimes \gls{author} is also
    used}
}

\newglossaryentry{free-software}{
    name={free software},
    description={software licensed under a \gls{foss} \gls{license}}
}

\newglossaryentry{open-source}{
    name={open source},
    description={software licensed under a \gls{foss} \gls{license}}
}

\newglossaryentry{inbound}{
    name={inbound},
    description={third-party \glspl{license} that enter a first-party project}
}

\newglossaryentry{outbound}{
    name={outbound},
    description={\glspl{license} that are distributed to third parties along with a
    product}
}

\newglossaryentry{upstream}{
    name={upstream},
    description={anything in the supply chain that is closer to the original authors or maintainers of software; or the original authors or maintainers}
}
